\documentclass[11pt,letterpaper]{article}
\usepackage[lmargin=1in,rmargin=1in,tmargin=1in,bmargin=1in]{geometry}

% -------------------
% Packages - Packages to Use
% -------------------
\usepackage{
	amsmath,			% Math Environments
	amssymb,			% Extended Symbols
	enumerate,		    % Enumerate Environments
	graphicx,			% Include Images
	lastpage,			% Reference Lastpage
	multicol,			% Use Multi-columns
	multirow,			% Use Multi-rows
	bm
}


% -------------------
% Font
% -------------------
\usepackage[T1]{fontenc}
\usepackage{charter}


% -------------------
% Heading Commands
% -------------------
\newcommand{\class}{Multivariable Calculus}
\newcommand{\term}{Trimester 2, 2020}
\newcommand{\instructor}{Samuel Leitermann-Long}
\newcommand{\head}[2]{%
\thispagestyle{empty}
\vspace*{-0.5in}
\noindent\begin{tabular*}{\textwidth}{l @{\extracolsep{\fill}} r @{\extracolsep{6pt}} l}
	\textbf{#1} & \textbf{Name:} & \makebox[8cm]{\hrulefill} \\
	\textbf{#2} & & \\
	\textbf{\class:\; \term} & & \\
	\textbf{Instructor: \instructor}
\end{tabular*} \\
\rule[2ex]{\textwidth}{2pt} %
}


% -------------------
% Commands
% -------------------
\newcommand{\prob}{\noindent\textbf{Problem. }}
\newcounter{problem}
\newcommand{\problem}{
	\stepcounter{problem}%
	\noindent \textbf{Problem \theproblem. }%
}
\newcommand{\pointproblem}[1]{
	\stepcounter{problem}%
	\noindent \textbf{Problem \theproblem.} (#1 points)\,%
}
\newcommand{\pspace}{\par\vspace{\baselineskip}}
\newcommand{\ds}{\displaystyle}


% -------------------
% Header & Footer
% -------------------
\usepackage{fancyhdr}

\fancypagestyle{pages}{
	%Headers
	\fancyhead[L]{}
	\fancyhead[C]{}
	\fancyhead[R]{}
\renewcommand{\headrulewidth}{0pt}
	%Footers
	\fancyfoot[L]{}
	\fancyfoot[C]{}
	\fancyfoot[R]{}
\renewcommand{\footrulewidth}{0.0pt}
}
\headheight=0pt
\footskip=14pt

\pagestyle{pages}


% -------------------
% Content
% -------------------
\begin{document}
\head{12/02/2020}
\textbf{Instructions:} \par \noindent This assignment is to help you become more familiar with 3D space.  \pspace


% Question
\prob Find all values of $a$ such that the vectors $<2, 4, a>$ and $<0, -1, a>$ are orthogonal. \pspace


% Question
\prob Find the equation of the plane that passes through the point $(1, 4, 3)$ and contains the line given by $x=\frac{y-1}{2} = z+1.$  \pspace


% Question
\prob Consider the two planes given by $x+y+2z = 0$ and $2x - y + 3z = 0$. Brainstorm a method to find the line of intersection between the two planes. Give the parametric and symmetric equations of that line.  \pspace


% Question
\prob Consider the vectors $a = <2,1,-9>$, $b = <-1,2,0>$, and $c = <4,-2,1>$. Find the following quantities:
\pspace
a.) b x c \pspace
b.) b x a \pspace
c.) a x b \pspace
d.) a x b x c \pspace

% Question
\prob Consider a boat crossing a river that runs east to west (the boat is traveling south to north.) The boat velocity is 5 km/h due north in still water and the water has a current of 2 km/h due west. What is the velocity of the boat relative to the shore? What is the angle $\theta$ that the boat's actual trajectory makes with due north?
\pspace

% Question
\prob Find the volume of the parallelpiped formed by $a = <1,4,1>$, $b = <3,6,2>$, and $c = <-2,1,-5>.$   \pspace

\end{document}
